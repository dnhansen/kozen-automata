% Document setup
\documentclass[article, a4paper, 11pt, oneside]{memoir}
\usepackage[utf8]{inputenc}
\usepackage[T1]{fontenc}
\usepackage[UKenglish]{babel}

% Document info
\newcommand\doctitle{Kozen, \emph{Automata and Computability}}
\newcommand\docauthor{Danny Nygård Hansen}

% Formatting and layout
\usepackage[autostyle]{csquotes}
\usepackage[final]{microtype}
\usepackage{xcolor}
\frenchspacing
\usepackage{latex-sty/articlepagestyle}
\usepackage{latex-sty/articlesectionstyle}
% \usepackage{latex-sty/amalgsymbol}

% Fonts
\usepackage[largesmallcaps]{kpfonts}
\DeclareSymbolFontAlphabet{\mathrm}{operators} % https://tex.stackexchange.com/questions/40874/kpfonts-siunitx-and-math-alphabets
\linespread{1.06}
\let\mathfrak\undefined
\usepackage{eufrak}
\usepackage{inconsolata}
% \usepackage{amssymb}

% Hyperlinks
\usepackage{hyperref}
\definecolor{linkcolor}{HTML}{4f4fa3}
\hypersetup{%
	pdftitle=\doctitle,
	pdfauthor=\docauthor,
	colorlinks,
	linkcolor=linkcolor,
	citecolor=linkcolor,
	urlcolor=linkcolor,
	bookmarksnumbered=true
}

% Equation numbering
\numberwithin{equation}{chapter}

% Footnotes
\footmarkstyle{\textsuperscript{#1}\hspace{0.25em}}

% Mathematics
\usepackage{latex-sty/basicmathcommands}
\usepackage{latex-sty/framedtheorems}
\usepackage{tikz-cd}
\tikzcdset{arrow style=math font} % https://tex.stackexchange.com/questions/300352/equalities-look-broken-with-tikz-cd-and-math-font
\usetikzlibrary{babel}
\usepackage{bm}

% Lists
\usepackage{enumitem}
\setenumerate[0]{label=\normalfont(\alph*)}
\setlist{  
  listparindent=\parindent,
  parsep=0pt,
}

% Bibliography
\usepackage[backend=biber, style=authoryear, maxcitenames=2, useprefix]{biblatex}
\addbibresource{references.bib}

% Title
\title{\doctitle}
\author{\docauthor}

%% Framed exercise environment

\mdfdefinestyle{swannexercise}{%
    skipabove=0.5em plus 0.4em minus 0.2em,
	skipbelow=0.5em plus 0.4em minus 0.2em,
	leftmargin=-5pt,
	rightmargin=-5pt,
	innerleftmargin=5pt,
	innerrightmargin=5pt,
	innertopmargin=5pt,
	innerbottommargin=4pt,
	linewidth=0pt,
	splittopskip=1.2em minus 0.2em,
	splitbottomskip=0.5em plus 0.2em minus 0.1em,
	backgroundcolor=backgroundcolor,
	frametitlebackgroundcolor=titlecolor,
	frametitlefont={\scshape},
    theoremseparator={},
    % theoremspace={},
	frametitleaboveskip=3pt,
	frametitlebelowskip=2pt
}

\mdtheorem[style=swannexercise]{exerciseframed}{Exercise}

\let\oldexerciseframed\exerciseframed
\renewcommand{\exerciseframed}{%
    \crefalias{theorem}{exerciseframed}%
    \oldexerciseframed}

\settocdepth{subsection}
\renewenvironment{exerciseframed}[1][]{%
    \refstepcounter{exerciseframed}%
    \addcontentsline{toc}{subsection}{Exercise #1}%
    \begin{exerciseframed*}[#1]%
    \label{ex:#1}%
}{%
    \end{exerciseframed*}%
}

\newcommand{\exref}[1]{%
    \hyperref[ex:#1]{Exercise~\mylabel}%
}

\theoremstyle{nonumberplain}
\theoremsymbol{\ensuremath{\square}}
\newtheorem{solution}{Solution}

\let\oldsolution\solution
\renewcommand{\solution}{%
  \crefalias{theorem}{solution}%
  \oldsolution}

\newcommand{\solutionlabelfont}[1]{{\normalfont\color{linkcolor}#1}}
\newlist{solutionsec}{enumerate}{1}
\setlist[solutionsec]{leftmargin=0pt, parsep=0pt, listparindent=\parindent, font=\solutionlabelfont, label=(\alph*), labelsep=0pt, labelwidth=20pt, itemindent=20pt, align=left, itemsep=10pt}

\newenvironment{displaytheorem}{%
	\begin{displayquote}\itshape%
}{%
	\end{displayquote}%
}


% Redefine environments to support lack of section numbering
\theoremstyle{myexample}
\renewtheorem{remark}{Remark}

\theoremstyle{myexamplebreak}
\renewtheorem{remarkbreak}{Remark}


\begin{document}

\maketitle

\setcounter{secnumdepth}{-2}
\chapter{Miscellaneous Exercises}

\section{Finite Automata and Regular Sets}

\newcommand{\calL}{\mathcal{L}}

\begin{remarkbreak}[The Pumping Lemma]
    We prove the pumping lemma:
    %
    \begin{displaytheorem}
        Let $A$ be a regular set over an alphabet $\Sigma$. Then there exists a $k \in \naturals$ such that for any $x,y,z \in \Sigma^*$ with $xyz \in A$ and $\card{y} \geq k$, there exist $u,v,w \in \Sigma^*$ such that $y = uvw$, $v \neq \epsilon$, and $xuv^n wz \in A$ for all $n \in \naturals$.
    \end{displaytheorem}
    %
    Let $M = (Q, \Sigma, \delta, s, F)$ be a DNF accepting $A$, let $k = \card{Q}$, and let $x,y,z$ be given as above. Write $y = y_1 \cdots y_k$ with $y_i \in \Sigma$. Let $p_0 = \hat\delta(s,x)$ and define $p_i = \delta(p_{i-1},y_i)$ for $i = 1, \ldots, k$. Then since $k+1 > \card{Q}$, there exist $i < j$ such that $p_i = p_j$. Let $u = y_1 \cdots y_i$, $v = y_{i+1} \cdots y_j$ and $w = y_{j+1} \cdots y_k$, and notice that $v \neq \epsilon$ since $i < j$. Furthermore, clearly $\hat\delta(p_i,v) = p_j = p_i$, so by induction we have
    %
    \begin{equation*}
        \hat\delta(p_i, v^n)
            = \hat\delta \bigl( \hat\delta(p_i, v^{n-1}), v \bigr)
            = \hat\delta(p_i, v)
            = p_i.
    \end{equation*}
    %
    Since $\hat\delta(s,xu) = p_i$, it follows that
    %
    \begin{equation*}
        \hat\delta(s,xuv^n wz)
            = \hat\delta \bigl( \hat\delta(p_i, v^n), wz \bigr)
            = \hat\delta(p_i, vwz)
            = \hat\delta(s, xyz)
            \in F.
    \end{equation*}
\end{remarkbreak}

\begin{exerciseframed}[7]
    If $\Sigma$ is an alphabet, consider the triple $(\powerset{\Sigma^*}, \union, \cdot)$, where $\cdot$ denotes concatenation. Since contatenation distributes over unions, this is a semiring\footnotemark{} with addition $\union$ and multiplication $\cdot$. In particular, the set of matrices with entries in $\powerset{\Sigma^*}$ inherits an addition and multiplication, making it into a semiring itself.

    Given a DFA $M = (Q, \Sigma, \delta, s, F)$ with $n$ states, define its \emph{transition matrix} $G$ as the $n \times n$ matrix given by
    %
    \begin{equation*}
        G_{uv}
            = \set{a \in \Sigma}{\delta(u,a) = v}
    \end{equation*}
    %
    for $u,v \in Q$.
    %
    \begin{enumerate}
        \item Prove that
        %
        \begin{equation*}
            (G^n)_{uv}
                = \set{x \in \Sigma^*}{ \text{$\card{x} = n$, and $\hat\delta(u,x) = v$}}.
        \end{equation*}

        \item Define the Kleene closure $A^*$ of the matrix $A$ to be the componentwise union of all the powers of $A$:
        %
        \begin{equation*}
            (A^*)_{uv}
                = \bigunion_{n \in \naturals} (A^n)_{uv}.
        \end{equation*}
        %
        Prove that
        %
        \begin{equation*}
            \calL(M)
                = \bigunion_{t \in F} (G^*)_{st}.
        \end{equation*}
    \end{enumerate}
\end{exerciseframed}
\footnotetext{Recall that, in a semiring, we require that multiplication by the additive identity annihilate any element in the semiring. In a ring this follows from additive cancellation. The set $\powerset{\Sigma^*}$ is in fact a Kleene algebra, but we do not need this fact below.}

\begin{solution}
\begin{solutionsec}
    \item For $n = 0$ we have $x = \epsilon$, so $\hat\delta(u,\epsilon) = v$ if and only if $u = v$, and so the right-hand side above equals $I_{uv}$.
    
    Assume that the claim holds for some $n \in \naturals$. Then
    %
    \begin{equation*}
        (G^{n+1})_{uv}
            = (G^n G)_{uv}
            = \bigunion_{w \in Q} (G^n)_{uw} G_{wv}.
    \end{equation*}
    %
    Now notice that, by induction,
    %
    \begin{align*}
        (G^n)_{uw} G_{wv}
            &= \set{x \in \Sigma^*}{ \text{$\card{x} = n$, and $\hat\delta(u,x) = w$}} G_{wv} \\
            &= \set{xa \in \Sigma^*}{ \text{$\card{x} = n$, $\hat\delta(u,x) = w$, and $\delta(w,a) = v$}} \\
            &= \set{xa \in \Sigma^*}{ \text{$\card{x} = n$, $\hat\delta(u,x) = w$, and $\delta(\hat\delta(u,x),a) = v$}} \\
            &= \set{xa \in \Sigma^*}{ \text{$\card{x} = n$, $\hat\delta(u,x) = w$, and $\hat\delta(u,xa) = v$}}.
    \end{align*}
    %
    Since the condition $\hat\delta(u,x) = w$ is always satisfied for some $w \in Q$, it follows that
    %
    \begin{align*}
        \bigunion_{w \in Q} (G^n)_{uw} G_{wv}
            &= \bigunion_{w \in Q} \set{xa \in \Sigma^*}{ \text{$\card{x} = n$, $\hat\delta(u,x) = w$, and $\hat\delta(u,xa) = v$}} \\
            &= \set{xa \in \Sigma^*}{ \text{$\card{xa} = n+1$, and $\hat\delta(u,xa) = v$}} \\
            &= (G^{n+1})_{uv}
    \end{align*}
    %
    as desired.

    \item A string $x \in \Sigma^*$ lies in $\calL(M)$ if and only if there is a $t \in F$ such that $\hat\delta(s,x) = t$, iff $x \in (G^{\card{x}})_{st}$. The claim follows easily.
\end{solutionsec}
\end{solution}


\newcommand{\Bepsilon}{\bm{\epsilon}}

\begin{exerciseframed}[10]
    Recall from Lecture 10 that an \emph{$\epsilon$-NFA} is a structure
    %
    \begin{equation*}
        M = (Q, \Sigma, \Bepsilon, \Delta, S, F)
    \end{equation*}
    %
    such that $\Bepsilon$ is a special symbol not in $\Sigma$ and
    %
    \begin{equation*}
        M_{\Bepsilon} = (Q, \Sigma \union \{\Bepsilon\}, \Delta, S, F)
    \end{equation*}
    %
    is an ordinary NFA.
    
    Define the \emph{$\epsilon$-closure} $C_{\Bepsilon}(A)$ of a set $A \subseteq Q$ to be the set of all states reachable from some state in $A$ under a sequence of zero or more $\epsilon$-transitions:
    %
    \begin{equation*}
        C_{\Bepsilon}(A)
            = \bigunion_{x \in \{\Bepsilon\}^*} \hat\Delta(A,x)
            = \bigunion_{k \in \naturals} \hat\Delta(A, \Bepsilon^k).
    \end{equation*}

    \begin{enumerate}
        \item Using $\epsilon$-closure, define formally acceptance for $\epsilon$-NFAs in a way that captures the intuitive description given in Lecture 6.
        
        \item Prove that under your definition, $\epsilon$-NFAs accept only regular sets.

        \item Prove that the two definitions of acceptance -- the one given in part (a) involving $\epsilon$-closure and the one given in Lecture 10 involving homomorphisms -- are equivalent.
    \end{enumerate}
\end{exerciseframed}

\begin{solution}
\begin{solutionsec}
    \item Notice that $\epsilon$-closure is a map $C_{\Bepsilon} \colon \powerset{Q} \to \powerset{Q}$, and that since $M_{\Bepsilon}$ is an NFA, $\Delta$ is a map $Q \prod (\Sigma \union \{\Bepsilon\}) \to \powerset{Q}$. We define a modified transition function
    %
    \begin{equation*}
        \Delta_{\Bepsilon} \colon Q \prod (\Sigma \union \{\Bepsilon\}) \to \powerset{Q}
    \end{equation*}
    %
    by $\Delta_{\Bepsilon} = C_{\Bepsilon} \circ \Delta$. We then say that $M$ \emph{accepts} $x \in \Sigma^*$ if the NFA
    %
    \begin{equation*}
        (Q, \Sigma \union \{\Bepsilon\}, \Delta_{\Bepsilon}, C_{\Bepsilon}(S), F)
    \end{equation*}
    %
    accepts $x$. Denoting the corresponding extended transition function by $\hat\Delta_{\Bepsilon}$, this means that $M$ accepts $x$ if $\hat\Delta_{\Bepsilon}(C_{\Bepsilon}(S),x) \intersect F \neq \emptyset$.

    \item This is obvious from (a), since NFAs only accept regular sets by Theorem~6.4.

    \item First define
    %
    \begin{align*}
        \calL(M)
            &= h(\calL(M_{\Bepsilon})) \\
        \calL'(M)
            &= \set{x \in \Sigma^*}{\hat\Delta_{\Bepsilon}(C_{\Bepsilon}(S),x) \intersect F \neq \emptyset},
    \end{align*}
    %
    where $h \colon (\Sigma \union \{\Bepsilon\})^* \to \Sigma^*$ is given by $h(\Bepsilon) = \epsilon$ and $h(a) = a$ for $a \in \Sigma$. We prove the following claim:
    %
    \begin{displaytheorem}
        For every $A \subseteq Q$ and $x \in \Sigma^*$ we have
        %
        \begin{equation*}
            \bigunion_{y \in h\preim(x)} \hat\Delta(A,y)
                = \hat\Delta_{\Bepsilon}(C_{\Bepsilon}(A),x).
        \end{equation*}
    \end{displaytheorem}
    %
    The equality $\calL(M) = \calL'(M)$ follows easily by letting $A = S$. We prove the claim by induction on the length of $x$. If $x = \epsilon$, then
    %
    \begin{equation*}
        \hat\Delta_{\Bepsilon}(C_{\Bepsilon}(A),\epsilon)
            = C_{\Bepsilon}(A)
            = \bigunion_{k \in \naturals} \hat\Delta(A, \Bepsilon^k)
            = \bigunion_{y \in h\preim(\Bepsilon)} \hat\Delta(A,y),
    \end{equation*}
    %
    where the last equality follows since $h$ sends precisely strings on the form $\Bepsilon^k$ to $\epsilon$. Now assume that the claim holds for some string $x \in \Sigma^*$, and let $a \in \Sigma$. Then
    %
    \begin{align*}
        \hat\Delta_{\Bepsilon}(C_{\Bepsilon}(A), xa)
            &= \hat\Delta_{\Bepsilon} \bigl( \hat\Delta_{\Bepsilon}(C_{\Bepsilon}(A), x), a \bigr) \\
            &= \hat\Delta_{\Bepsilon} \bigl( C_{\Bepsilon} \circ \hat\Delta_{\Bepsilon}(C_{\Bepsilon}(A), x), a \bigr) \\
            &= \hat\Delta_{\Bepsilon} \biggl( C_{\Bepsilon} \Bigl( \bigunion_{y \in h\preim(x)} \hat\Delta(A,y) \Bigr), a \biggr) \\
            &= \bigunion_{b \in h\preim(a)} \hat\Delta \biggl( \bigunion_{y \in h\preim(x)} \hat\Delta(A,y), b \biggr) \\
            &= \bigunion_{b \in h\preim(a)} \bigunion_{y \in h\preim(x)} \hat\Delta \bigl( \hat\Delta(A,y), b \bigr) \\
            &= \bigunion_{b \in h\preim(a)} \bigunion_{y \in h\preim(x)} \hat\Delta(A,yb) \\
            &= \bigunion_{z \in h\preim(xa)} \hat\Delta(A,z),
    \end{align*}
    %
    proving the claim.
\end{solutionsec}
\end{solution}

\end{document}